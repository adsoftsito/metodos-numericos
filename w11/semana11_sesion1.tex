% Author: Adolfo Centeno 
% Waves Lab

 
\documentclass{beamer}
\setbeamertemplate{navigation symbols}{}
\usepackage[utf8]{inputenc}
\usepackage{beamerthemeshadow}
\usepackage{listings}
\usepackage{hyperref}

\hypersetup{
  colorlinks=true,
  linkcolor=blue!50!red,
  urlcolor=green!70!black
}

\begin{document}
\title{ITESM}  
\subtitle{Campus Puebla\\METODOS NUMERICOS EN INGENIERIA
}
\author{Adolfo Centeno}
\date{\today} 


\begin{frame}
\titlepage
\end{frame}

\begin{frame}\frametitle{Table of contents}
\tableofcontents
\end{frame} 


\section{W11 - Sesion 1 }

\begin{frame}

\textbf{W11 Sesion 1 - Actividades:}

\begin{enumerate}
\item
	 \href{https://github.com/adsoftsito/metodos-numericos/blob/master/w11/reglineal.pdf}{Conceptos de regresion lineal (pdf)}.

\item
	Resolver en excel ejercicio de regresion lineal

\end{enumerate} 

\end{frame}


\section{W11 Sesion 1 - Tareas }

\begin{frame}


\textbf{W11 Sesion 1 - Tareas}


\begin{enumerate}
\item

Resolver en Excel ejercicios adicionales de regresion lineal de pagina 3, comprobar resultados con tabla de pagina 4.


\end{enumerate} 


\end{frame}




\end{document}
