% Author: Adolfo Centeno 
% Waves Lab

 
\documentclass{beamer}
\setbeamertemplate{navigation symbols}{}
\usepackage[utf8]{inputenc}
\usepackage{beamerthemeshadow}
\usepackage{listings}
\usepackage{hyperref}

\hypersetup{
  colorlinks=true,
  linkcolor=blue!50!red,
  urlcolor=green!70!black
}

\begin{document}
\title{ITESM}  
\subtitle{Campus Puebla\\METODOS NUMERICOS EN INGENIERIA
}
\author{Adolfo Centeno}
\date{\today} 


\begin{frame}
\titlepage
\end{frame}

\begin{frame}\frametitle{Table of contents}
\tableofcontents
\end{frame} 


\section{W11 - Sesion 1 }

\begin{frame}

\textbf{W12 Sesion 1 - Actividades:}

\begin{enumerate}
\item
	 Polinomio de newthon \href{https://www.youtube.com/watch?v=WgZvz57c5CQ}{Valores de Xi Equidistantes}.

\item
	Resolver en Excel  ejercicio de polinomio de newton de segundo grado. \\
	
	xi	yi \\
-4	4 \\
-1	1 \\
2	5

\item Comprobar con código de \href{https://www.youtube.com/watch?v=WgZvz57c5CQ}{polinomio de newthon} en Matlab
\end{enumerate} 

\end{frame}


\section{W12 Sesion 1 - Tareas }

\begin{frame}


\textbf{W12 Sesion 1 - Tareas}


\begin{enumerate}
\item

Resolver el polinomonio de 2o grado para los siguientes puntos. \\

xi	yi \\
100	1.6729 \\
105	1.4194 \\
110	1.2102

\item
	Comprobar en Matlab


\end{enumerate} 


\end{frame}




\end{document}
