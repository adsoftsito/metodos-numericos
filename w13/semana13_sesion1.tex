% Author: Adolfo Centeno 
% Waves Lab

 
\documentclass{beamer}
\setbeamertemplate{navigation symbols}{}
\usepackage[utf8]{inputenc}
\usepackage{beamerthemeshadow}
\usepackage{listings}
\usepackage{hyperref}

\hypersetup{
  colorlinks=true,
  linkcolor=blue!50!red,
  urlcolor=green!70!black
}

\begin{document}
\title{ITESM}  
\subtitle{Campus Puebla\\METODOS NUMERICOS EN INGENIERIA
}
\author{Adolfo Centeno}
\date{\today} 


\begin{frame}
\titlepage
\end{frame}

\begin{frame}\frametitle{Table of contents}
\tableofcontents
\end{frame} 


\section{W13 - Sesion 1 }

\begin{frame}

\textbf{W13 Sesion 1 - Actividades:}

\begin{enumerate}
\item
	 Polinomio de newthon para puntos NO Equidistantes

\item
	Resolver en Excel  ejercicio de polinomio de newton de tercer grado. \\
	
xi	yi \\
1	4 \\
2	3.5 \\
3	4 \\
5   5.6

\item Comprobar con código de \href{https://github.com/adsoftsito/metodos-numericos/blob/master/w12/polinomionewthon/polnewton.pdf}{polinomio de newthon} en Matlab

\item Integracion numerica \href{https://github.com/adsoftsito/metodos-numericos/blob/master/w13/integracion_numerica.pdf}{conceptos} 



\end{enumerate} 

\end{frame}


\section{W13 Sesion 1 - Tareas }

\begin{frame}


\textbf{W13 Sesion 1 - Tareas}


\begin{enumerate}
\item

Resolver el polinomonio de 2o grado del siguiente \href{https://www.youtube.com/watch?v=AISHH6goWUs}{video}


\item
	Comprobar en Matlab


\end{enumerate} 


\end{frame}




\end{document}
