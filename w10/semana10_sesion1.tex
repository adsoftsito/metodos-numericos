% Author: Adolfo Centeno 
% Waves Lab

 
\documentclass{beamer}
\setbeamertemplate{navigation symbols}{}
\usepackage[utf8]{inputenc}
\usepackage{beamerthemeshadow}
\usepackage{listings}
\usepackage{hyperref}

\hypersetup{
  colorlinks=true,
  linkcolor=blue!50!red,
  urlcolor=green!70!black
}

\begin{document}
\title{ITESM}  
\subtitle{Campus Puebla\\METODOS NUMERICOS EN INGENIERIA
}
\author{Adolfo Centeno}
\date{\today} 


\begin{frame}
\titlepage
\end{frame}

\begin{frame}\frametitle{Table of contents}
\tableofcontents
\end{frame} 


\section{W10 - Sesion 1 }

\begin{frame}

\textbf{W10 Sesion 1 - Actividades:}

\begin{enumerate}
\item
	\href{https://es.wikipedia.org/wiki/Metodo_de_Jacobi}{Metodo Newthon-Raphson} para ecuaciones NO lineales: \href{https://www.youtube.com/watch?v=QS3wqOQabVY}{Video Ec. 2x2}.\\Analizar y ejecutar el código en Matlab, modificar para ingresar sistema de ecuaciones F y la matriz Jacobiana J y datos de entrada X	

\item
	\href{https://github.com/adsoftsito/metodos-numericos/blob/master/w10/newthon-raphson/newton_raphson_2.m}{Codigo Newthon-Raphson Ec 2x2} 
	

\end{enumerate} 

\end{frame}


\section{W9 Sesion 1 - Tareas }

\begin{frame}


\textbf{W9 Sesion 1 - Tareas}


\begin{enumerate}
\item
	Resolver por Jacobi y Seidel: \href{https://www.youtube.com/watch?v=lEtDw4HUAmY&t=858s}{ejercicio 1}.	
\item
	Resolver por Jacobi y Seidel: \href{https://www.youtube.com/watch?v=k2dB1IVWTng}{ejercicio 2}.	

\end{enumerate} 


\end{frame}




\end{document}
