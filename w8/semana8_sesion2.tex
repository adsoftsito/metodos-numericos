% Author: Adolfo Centeno 
% Waves Lab

 
\documentclass{beamer}
\setbeamertemplate{navigation symbols}{}
\usepackage[utf8]{inputenc}
\usepackage{beamerthemeshadow}
\usepackage{listings}
\usepackage{hyperref}

\hypersetup{
  colorlinks=true,
  linkcolor=blue!50!red,
  urlcolor=green!70!black
}

\begin{document}
\title{ITESM}  
\subtitle{Campus Puebla\\METODOS NUMERICOS EN INGENIERIA
}
\author{Adolfo Centeno}
\date{\today} 


\begin{frame}
\titlepage
\end{frame}

\begin{frame}\frametitle{Table of contents}
\tableofcontents
\end{frame} 


\section{W8 - Sesion 1 }

\begin{frame}

\textbf{W8 Sesion 1 - Actividades:}

\begin{enumerate}
\item
    Latex templates:  \\
    - \href{https://miktex.org/download}{Paper}. \\
    - \href{https://github.com/adsoftsito/metodos-numericos/blob/master/w8/semana8_sesion1.tex}{Slides} 
\item
	Eliminacion Gausiana: \href{https://www.youtube.com/watch?v=XRcx8-2lLJI}{Video}.	
\item
	Gauss-Jordan: \href{https://www.youtube.com/watch?v=pUabaQqbrug}{Video}.	
\item
	\href{https://github.com/adsoftsito/metodos-numericos/blob/master/w8/eliminaciongausiana/eliminaciongausian.m}{Codigo Matlab} 

	

\end{enumerate} 

\end{frame}


\section{W8 Sesion 1 - Tareas }

\begin{frame}


\textbf{W8 Sesion 1 - Tareas}


\begin{enumerate}
\item
	Eliminacion Gausiana 4X4: \href{https://www.youtube.com/watch?v=uL3JwFy9BWA}{ejericio 1}.	
\item
	Gauss-Jordan 4x4 : \href{https://www.youtube.com/watch?v=uL3JwFy9BWA&t=497s}{ejercicio 2}.	

\end{enumerate} 


\end{frame}




\end{document}
