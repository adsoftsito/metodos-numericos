% Author: Adolfo Centeno 
% Waves Lab

 
\documentclass{beamer}
\setbeamertemplate{navigation symbols}{}
\usepackage[utf8]{inputenc}
\usepackage{beamerthemeshadow}
\usepackage{listings}
\usepackage{hyperref}

\hypersetup{
  colorlinks=true,
  linkcolor=blue!50!red,
  urlcolor=green!70!black
}

\begin{document}
\title{ITESM}  
\subtitle{Campus Puebla\\METODOS NUMERICOS EN INGENIERIA
}
\author{Adolfo Centeno}
\date{\today} 


\begin{frame}
\titlepage
\end{frame}

\begin{frame}\frametitle{Table of contents}
\tableofcontents
\end{frame} 


\section{W14 - Sesion 2 }

\begin{frame}

\textbf{W14 Sesion 2 - Actividades:}

\begin{enumerate}

\item
    Ver \href{https://es.symbolab.com/solver/ordinary-differential-equation-calculator}{Calculadora} en linea de ecuaciones diferenciales

\item
	Metodo de \href{https://gomez-metodos-numericos.webnode.es/ecuaciones-diferenciales-ordinarias/ralston/}{Ralston}

\item Resolver en Excel y usando el metodo de Ralston las siguiente funcion

y' =  3 x*x


\item
	Metodo de \href{https://www.youtube.com/watch?v=YCLUN-2EQB8}{Runge Kuta 4}

\item Resolver en Excel y usando el metodo de Runge Kuta 4 la siguiente funcion

y'  =  3 x*x 



\end{enumerate} 

\end{frame}


\section{W14 Sesion 2 - Tareas }

\begin{frame}


\textbf{W14 Sesion 2 - Tareas}


\begin{enumerate}
\item Resolver usando Ralston el siguiente ejercicio

 
\href{https://gomez-metodos-numericos.webnode.es/ecuaciones-diferenciales-ordinarias/ralston/}{Ralston} \\


\item Resolver usando Runge Kuta 4 el siguiente ejercicio

 
\href{https://www.youtube.com/watch?v=RR_VprIzSGM}{Runge Kuta 4} \\



\end{enumerate} 


\end{frame}




\end{document}
