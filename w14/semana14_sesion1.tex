% Author: Adolfo Centeno 
% Waves Lab

 
\documentclass{beamer}
\setbeamertemplate{navigation symbols}{}
\usepackage[utf8]{inputenc}
\usepackage{beamerthemeshadow}
\usepackage{listings}
\usepackage{hyperref}

\hypersetup{
  colorlinks=true,
  linkcolor=blue!50!red,
  urlcolor=green!70!black
}

\begin{document}
\title{ITESM}  
\subtitle{Campus Puebla\\METODOS NUMERICOS EN INGENIERIA
}
\author{Adolfo Centeno}
\date{\today} 


\begin{frame}
\titlepage
\end{frame}

\begin{frame}\frametitle{Table of contents}
\tableofcontents
\end{frame} 


\section{W14 - Sesion 1 }

\begin{frame}

\textbf{W14 Sesion 1 - Actividades:}

\begin{enumerate}

\item
	Repaso de \href{https://www.youtube.com/watch?v=rd2jKGQJucE}{Ecuaciones diferenciales}

\item
    Ver \href{https://es.symbolab.com/solver/ordinary-differential-equation-calculator}{Calculadora} en linea de ecuaciones diferenciales

\item
	Metodo de \href{https://www.youtube.com/watch?v=V6wLYLvqZ84}{Euler}

\item Resolver en Excel y usando el metodo de Euler las siguientes funciones

y' =  3 x*x

	
\item Comprobar con \href{https://github.com/adsoftsito/metodos-numericos/blob/master/w14/euler.m}{codigo de euler} en Matlab

\item
	Metodo de \href{https://www.youtube.com/watch?v=mE11yv_zQKE}{Heun}

\item Resolver en Excel y usando el metodo de Heun las siguientes funciones

y'  =  3 x*x 


\item Comprobar con \href{https://github.com/adsoftsito/metodos-numericos/blob/master/w14/heun.m}{codigo de heun} en Matlab



\end{enumerate} 

\end{frame}


\section{W14 Sesion 1 - Tareas }

\begin{frame}


\textbf{W14 Sesion 1 - Tareas}


\begin{enumerate}
\item Resolver usando Euler y Heun las siguientes ecuaciones

 
\href{https://www.youtube.com/watch?v=lob94xNqq0w}{y =  2y - 1} \\

 
\href{https://www.youtube.com/watch?v=RR_VprIzSGM}{y = 0.1x – 3 * raiz(y)} \\


\href{https://www.youtube.com/watch?v=Ja9n0XLm3ww}{y = xy + xy*y}
 

\item
	Comprobar en Matlab


\end{enumerate} 


\end{frame}




\end{document}
